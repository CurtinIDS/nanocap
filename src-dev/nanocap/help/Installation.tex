\section{Installation}

There are three approaches to using \nanocap:

\begin{enumerate}
 \item As a standalone application.
 \item From source without rendering/GUI capabilities
 \item From source with rendering/GUI capabilities
\end{enumerate}

The installation procedures involved in each of the options above vary with increasing complexity yet this is balanced with an increase in versatility. For example, \nanocap~ compiled from source with with rendering and GUI capabilities can be used in parallelised code to produce and visualise multiple structures.


\subsection{Requirements}

\begin{enumerate}

\item As a standalone application.
 
\nanocap~ is built into a DMG for OSX and a .EXE for Windows. The required libraries listed below are bundled but not modified in line with the associated licenses. 

\textbf{OSX}:  \nanocap~works \textit{straight out of the box} simply extract the application from the DMG and drag it into the \textbf{Applications} folder.

\textbf{Windows}:  \nanocap~requires the Microsoft Visual C++ 2008 Redistributable Package which can be obtained from:

 \url{http://www.microsoft.com/en-au/download/details.aspx?id=29}


 \item From source without rendering/GUI capabilities

\begin{itemize}
 \item NumPy - Version 1.6.2
 \item Scipy - Version 0.11.0
 \item sqlite3 Version 2.6.0 (bundled with Python)
 \item C compiler (e.g. GCC)
 \item Fortran compiler (e.g. GFortran)
 \end{itemize}
 
 \item From source with rendering/GUI capabilities

\begin{itemize}
 \item Qt - Version 4.8.5 
 \item PySide - Version 1.1.1 (depends on Qt)
 \item VTK - Version 5.8 (+ Python Wrappers)
 \end{itemize}
 
 \end{enumerate}
Installation of the dependencies above is platform dependent and there are multiple methods of achieving the required python working environment. The simplest options are using package managers or binary distributions, i.e:
\newline\newline
OSX:
\begin{itemize}
 \item Homebrew \url{http://brew.sh}
 \item MacPorts \url{http://www.macports.org}
 \item Fink \url{http://www.finkproject.org/}
 \end{itemize}
Windows:
\begin{itemize}
 \item PythonXY \url{http://code.google.com/p/pythonxy}
 \item Enthought Python \url{http://www.enthought.com/products/epd}
 \end{itemize}
 Linux:
 \begin{itemize}
\item apt-get 
\item YUM (RPM Package Manager)
 \end{itemize}
 
\subsection{Installing from source}

After obtaining and installing the previously outlined requirements, a tar ball of \nanocap~can be downloaded from:

\url{http://sourceforge.net/projects/nanocap/files/src/}

As an alternative, the latest release can be checked out from the \texttt{mercurial} repository:
\begin{lstlisting}[keywordstyle={\color{myterminalfont}},
			language={sh},
			commentstyle={\it\color{myterminalfont}},
			emphstyle={\ttb\color{myterminalfont}},
			stringstyle={\color{myterminalfont}},
			showstringspaces={false},
			otherkeywords={self},
			emph={MyClass,__init__},
			frame={},
			basicstyle={\ttm\color{myterminalfont}},
			morekeywords={True,False},
			captionpos={b},
			backgroundcolor={\color{myterminalbg}}]		
bash-3.2$ hg clone http://hg.code.sf.net/p/nanocap/code-0 nanocap-code-0
\end{lstlisting}
After download and unpacking (if required), installation proceeds in the typically python fashion:
\begin{lstlisting}[keywordstyle={\color{myterminalfont}},
			language={sh},
			commentstyle={\it\color{myterminalfont}},
			emphstyle={\ttb\color{myterminalfont}},
			stringstyle={\color{myterminalfont}},
			showstringspaces={false},
			otherkeywords={self},
			emph={MyClass,__init__},
			frame={},
			basicstyle={\ttm\color{myterminalfont}},
			morekeywords={True,False},
			captionpos={b},
			backgroundcolor={\color{myterminalbg}}]		
bash-3.2$ python setup.py install
\end{lstlisting}

The installation runs a configure script (generated by \texttt{autoconf}) that detects available C and Fortran compilers and builds the associated shared libraries. To test for a successful installation, simply attempt to import \nanocap~in a python terminal. 

\begin{lstlisting}[keywordstyle={\color{myterminalfont}},
			language={sh},
			commentstyle={\it\color{myterminalfont}},
			emphstyle={\ttb\color{myterminalfont}},
			stringstyle={\color{myterminalfont}},
			showstringspaces={false},
			otherkeywords={self},
			emph={MyClass,__init__},
			frame={},
			basicstyle={\ttm\color{myterminalfont}},
			morekeywords={True,False},
			captionpos={b},
			backgroundcolor={\color{myterminalbg}}]		
bash-3.2$ python
Python 2.7.3 (default, Jul 19 2012, 13:57:53) 
[GCC 4.2.1 Compatible Apple Clang 3.1 (tags/Apple/clang-318.0.61)] on darwin
Type "help", "copyright", "credits" or "license" for more information.
>>> import nanocap
>>> print nanocap.__version__
1.0b9
\end{lstlisting}

The successful import and printing of the version number indicates the libraries installed correctly. To test the installation further try running the example scripts shown in Section \ref{Examples}.

If the GUI and rendering libraries have been successfully installed then \nanocap can be ran either as an application or as libraries. To run the GUI enabled \nanocap simply type:
\begin{lstlisting}[keywordstyle={\color{myterminalfont}},
			language={sh},
			commentstyle={\it\color{myterminalfont}},
			emphstyle={\ttb\color{myterminalfont}},
			stringstyle={\color{myterminalfont}},
			showstringspaces={false},
			otherkeywords={self},
			emph={MyClass,__init__},
			frame={},
			basicstyle={\ttm\color{myterminalfont}},
			morekeywords={True,False},
			captionpos={b},
			backgroundcolor={\color{myterminalbg}}]		
bash-3.2$ nanocap
\end{lstlisting}

into the command prompt.



